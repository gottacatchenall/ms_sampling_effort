%!TEX TS-program = xelatex
\documentclass[11pt]{article}

\usepackage[english]{babel}

\usepackage{amsmath,amssymb,amsfonts}
\usepackage[utf8]{inputenc}
\usepackage[T1]{fontenc}
\usepackage{stix}
\usepackage[scaled]{helvet}
\usepackage[scaled]{inconsolata}

\usepackage{lastpage}

\usepackage{setspace}

\usepackage{ccicons}

\usepackage[hang,flushmargin]{footmisc}

\usepackage{geometry}

\setlength{\parindent}{0pt}
\setlength{\parskip}{6pt plus 2pt minus 1pt}

\usepackage{fancyhdr}
\renewcommand{\headrulewidth}{0pt}\providecommand{\tightlist}{%
  \setlength{\itemsep}{0pt}\setlength{\parskip}{0pt}}

\makeatletter
\newcounter{tableno}
\newenvironment{tablenos:no-prefix-table-caption}{
  \caption@ifcompatibility{}{
    \let\oldthetable\thetable
    \let\oldtheHtable\theHtable
    \renewcommand{\thetable}{tableno:\thetableno}
    \renewcommand{\theHtable}{tableno:\thetableno}
    \stepcounter{tableno}
    \captionsetup{labelformat=empty}
  }
}{
  \caption@ifcompatibility{}{
    \captionsetup{labelformat=default}
    \let\thetable\oldthetable
    \let\theHtable\oldtheHtable
    \addtocounter{table}{-1}
  }
}
\makeatother

\usepackage{array}
\newcommand{\PreserveBackslash}[1]{\let\temp=\\#1\let\\=\temp}
\let\PBS=\PreserveBackslash

\usepackage[breaklinks=true]{hyperref}
\hypersetup{colorlinks,%
citecolor=blue,%
filecolor=blue,%
linkcolor=blue,%
urlcolor=blue}
\usepackage{url}

\usepackage{caption}
\setcounter{secnumdepth}{0}
\usepackage{cleveref}

\usepackage{graphicx}
\makeatletter
\def\maxwidth{\ifdim\Gin@nat@width>\linewidth\linewidth
\else\Gin@nat@width\fi}
\makeatother
\let\Oldincludegraphics\includegraphics
\renewcommand{\includegraphics}[1]{\Oldincludegraphics[width=\maxwidth]{#1}}

\usepackage{longtable}
\usepackage{booktabs}

\usepackage{color}
\usepackage{fancyvrb}
\newcommand{\VerbBar}{|}
\newcommand{\VERB}{\Verb[commandchars=\\\{\}]}
\DefineVerbatimEnvironment{Highlighting}{Verbatim}{commandchars=\\\{\}}
% Add ',fontsize=\small' for more characters per line
\usepackage{framed}
\definecolor{shadecolor}{RGB}{248,248,248}
\newenvironment{Shaded}{\begin{snugshade}}{\end{snugshade}}
\newcommand{\KeywordTok}[1]{\textcolor[rgb]{0.13,0.29,0.53}{\textbf{#1}}}
\newcommand{\DataTypeTok}[1]{\textcolor[rgb]{0.13,0.29,0.53}{#1}}
\newcommand{\DecValTok}[1]{\textcolor[rgb]{0.00,0.00,0.81}{#1}}
\newcommand{\BaseNTok}[1]{\textcolor[rgb]{0.00,0.00,0.81}{#1}}
\newcommand{\FloatTok}[1]{\textcolor[rgb]{0.00,0.00,0.81}{#1}}
\newcommand{\ConstantTok}[1]{\textcolor[rgb]{0.00,0.00,0.00}{#1}}
\newcommand{\CharTok}[1]{\textcolor[rgb]{0.31,0.60,0.02}{#1}}
\newcommand{\SpecialCharTok}[1]{\textcolor[rgb]{0.00,0.00,0.00}{#1}}
\newcommand{\StringTok}[1]{\textcolor[rgb]{0.31,0.60,0.02}{#1}}
\newcommand{\VerbatimStringTok}[1]{\textcolor[rgb]{0.31,0.60,0.02}{#1}}
\newcommand{\SpecialStringTok}[1]{\textcolor[rgb]{0.31,0.60,0.02}{#1}}
\newcommand{\ImportTok}[1]{#1}
\newcommand{\CommentTok}[1]{\textcolor[rgb]{0.56,0.35,0.01}{\textit{#1}}}
\newcommand{\DocumentationTok}[1]{\textcolor[rgb]{0.56,0.35,0.01}{\textbf{\textit{#1}}}}
\newcommand{\AnnotationTok}[1]{\textcolor[rgb]{0.56,0.35,0.01}{\textbf{\textit{#1}}}}
\newcommand{\CommentVarTok}[1]{\textcolor[rgb]{0.56,0.35,0.01}{\textbf{\textit{#1}}}}
\newcommand{\OtherTok}[1]{\textcolor[rgb]{0.56,0.35,0.01}{#1}}
\newcommand{\FunctionTok}[1]{\textcolor[rgb]{0.00,0.00,0.00}{#1}}
\newcommand{\VariableTok}[1]{\textcolor[rgb]{0.00,0.00,0.00}{#1}}
\newcommand{\ControlFlowTok}[1]{\textcolor[rgb]{0.13,0.29,0.53}{\textbf{#1}}}
\newcommand{\OperatorTok}[1]{\textcolor[rgb]{0.81,0.36,0.00}{\textbf{#1}}}
\newcommand{\BuiltInTok}[1]{#1}
\newcommand{\ExtensionTok}[1]{#1}
\newcommand{\PreprocessorTok}[1]{\textcolor[rgb]{0.56,0.35,0.01}{\textit{#1}}}
\newcommand{\AttributeTok}[1]{\textcolor[rgb]{0.77,0.63,0.00}{#1}}
\newcommand{\RegionMarkerTok}[1]{#1}
\newcommand{\InformationTok}[1]{\textcolor[rgb]{0.56,0.35,0.01}{\textbf{\textit{#1}}}}
\newcommand{\WarningTok}[1]{\textcolor[rgb]{0.56,0.35,0.01}{\textbf{\textit{#1}}}}
\newcommand{\AlertTok}[1]{\textcolor[rgb]{0.94,0.16,0.16}{#1}}
\newcommand{\ErrorTok}[1]{\textcolor[rgb]{0.64,0.00,0.00}{\textbf{#1}}}
\newcommand{\NormalTok}[1]{#1}

\newlength{\cslhangindent}
\setlength{\cslhangindent}{1.5em}
\newlength{\csllabelwidth}
\setlength{\csllabelwidth}{3em}
\newenvironment{CSLReferences}[3] % #1 hanging-ident, #2 entry spacing
 {% don't indent paragraphs
  \setlength{\parindent}{0pt}
  % turn on hanging indent if param 1 is 1
  \ifodd #1 \everypar{\setlength{\hangindent}{\cslhangindent}}\ignorespaces\fi
  % set entry spacing
  \ifnum #2 > 0
  \setlength{\parskip}{#2\baselineskip}
  \fi
 }%
 {}
\usepackage{calc} % for \widthof, \maxof
\newcommand{\CSLBlock}[1]{#1\hfill\break}
\newcommand{\CSLLeftMargin}[1]{\parbox[t]{\maxof{\widthof{#1}}{\csllabelwidth}}{#1}}
\newcommand{\CSLRightInline}[1]{\parbox[t]{\linewidth}{#1}}
\newcommand{\CSLIndent}[1]{\hspace{\cslhangindent}#1}\geometry{verbose,letterpaper,tmargin=2.5cm,bmargin=2.5cm,lmargin=2.5cm,rmargin=4.5cm}

\usepackage{lineno}
\usepackage[nolists,noheads]{endfloat}

\pagestyle{plain}

\doublespacing

\fancypagestyle{normal}
{
  \fancyhf{}
  \fancyfoot[R]{\footnotesize\sffamily\thepage\ of \pageref*{LastPage}}
}
\begin{document}
\thispagestyle{empty}
{\Large\bfseries\sffamily The missing link: discerning true from false
negatives when sampling species interaction networks}
\vskip 5em

%
\href{https://orcid.org/0000-0002-6506-6487}{Michael D.\,Catchen}%
%
\,\textsuperscript{1,2}\quad %
\href{https://orcid.org/0000-0002-0735-5184}{Timothée\,Poisot}%
%
\,\textsuperscript{3,2}\quad %
\href{https://orcid.org/0000-0002-6004-4027}{Laura\,Pollock}%
%
\,\textsuperscript{1,2}\quad %
\href{https://orcid.org/0000-0001-6075-8081}{Andrew\,Gonzalez}%
%
\,\textsuperscript{1,2}

\textsuperscript{1}\,McGill University\quad \textsuperscript{2}\,Québec
Centre for Biodiversity Sciences\quad \textsuperscript{3}\,Université de
Montréal


\textbf{Correspondance to:}\\
Michael D. Catchen --- \texttt{michael.catchen@mail.mcgill.ca}\\

\vfill
This work is released by its authors under a CC-BY 4.0 license\hfill\ccby\\
Last revision: \emph{\today}

\clearpage
\thispagestyle{empty}

\vfill
\textbf{\sffamily Abstract: }Species interactions and the networks that
emerge from them structure ecosystem processes and enable biodiversity
to persist through time. Still a robust understanding of interactions
between species, how human activity is effecting these intearctions, and
how these change will effect Earth's ecosystems in the future remains
elusive. This knowledge-gap is largely driven by a shortfall of
data---although species occurence data has rapidly increased in the last
decade, species interaction data has lagged behind, largely due to the
intrinsic difficulty of sampling interations. These sampling challenges
bias data. Here, we demonstrate that the realized false-negative rate
can be quite highly biased toward species with high relative abundance.
We then simulate observation on both 243 empirical food webs and
generated models to estimate the sampling effort required to reduce the
false-negative rate to less than 10\%. We then assess how false
negatives effect measurements of network properties and models of
network prediction. We conclude by discussing how understanding of
false-negatives can inform how we design sampling of species
interactions and the networks they form.
\vfill

\clearpage
\linenumbers
\pagestyle{normal}

\hypertarget{introduction}{%
\section{Introduction}\label{introduction}}

Understanding which and how species interact is both a fundamental
question of community ecology, but also an increasing imperative to
mitigate the consequences of human activity on biodiversity (Makiola et
al. 2020; Jordano 2016a) and to predict potential spillover of zoonotic
disease (Becker et al. 2021). Over the past decade biodiversity data has
become increasingly available. Modern remote-sensing has enabled
collection of data on spatial scales and resolutions previously
unimaginable, and improved in-situ sensing (Stephenson 2020) and
adoption of open data practices (Kenall, Harold, and Foote 2014) have
substantially amount of data available to ecologists. Still widespread
data about species \emph{interactions} remains elusive. Often observing
an interaction between two species requires human sampling, because
although remote methods can detect co-occurrence, this itself is not
necessarily indicative of interaction (Blanchet, Cazelles, and Gravel
2020). This constraint induces biases on species interaction data
subject to the spatial and temporal scales that humans can feasibly
sample.

\emph{Sampling effort} and its impact on the resulting data collected
from ecosystems has encouraged a long history of discourse. The recorded
number of species in a sample depends on the total number of
observations (Willott 2001; Walther et al. 1995), as do estimates of
population abundance (Griffiths 1998). This has motivated more
quantitatively robust approaches to account for error in sampling data
in many contexts: to determine if a given species is extinct (Boakes,
Rout, and Collen 2015), to determine sampling design (Moore and McCarthy
2016), and to measure global species richness (Carlson et al. 2020). In
the context of interactions, the initial concern was the compounding
effects of limited sampling effort combined with the amalgamation of
data (across both study sites and across taxonomic scales) could lead
any empirical set of observations to inadequately reflect the reality of
how species interact (Paine 1988). Martinez et al. (1999) showed that in
a plant-endophyte trophic network, network connectance is robust to
sampling effort, but this done in the context of a system for which
observation of 62,000 total interactions derived from 164,000
plant-stems was feasible. In some systems (e.g.~megafauna food-webs)
this many observations is either impractical or infeasible due to the
absolute abundance of the species in question.

Because we cannot feasibly observe all (or even most) interactions that
occur in nature, our samples end up capturing only a small fraction of
those interactions. This means we can be reasonably confident two
species actually interact if we have a record of it, but not at all
confident that two species \emph{do not} interact if we have no record
of those species observed together. In other words, we can't distinguish
true-negatives (two species \emph{never} interact) from
\emph{false-negatives} (two species interact in some capacity, but we
have not observed it). This is then amplified as the interaction data we
have is geographically toward the usual suspects (Poisot, Bergeron, et
al. 2021), This noise in data has practical consequences for answering
questions about species interactions (de Aguiar et al. 2019)---these
false-negatives could go on to effect the inferences we make about
network properties and relations among species, and our predictions
about how species will interact in the future.

This is compounded by semantic confusion about the definition of
``interaction.'' Here distinguish between: a species \emph{occurring}, a
species being \emph{observed occurring}, two species being observed
\emph{co-occurring}, and two species being observed \emph{interacting}
(fig.~\ref{fig:taxonomy}). In this manuscript, we refer to species
either as ``interacting''---two species co-occur (and, at least
sometimes, interact)---or ``not-interacting'' (two species that,
regardless of whether they co-occur, neither exhibits any meaningful
effect on the biomass of the other). In fig.~\ref{fig:taxonomy} we see
that, under our definition, observing two species co-occurring is a
prerequisite for observing an interaction between two species. But
species are not observed with equal probability but instead in
proportion to their relative biomass---you are much more likely to
observe a species of high relative abundance than one of very low
relative abundance (Poisot, Stouffer, and Gravel 2015). This assumes
that there are no associations in species co-occurrence due to an
interaction (perhaps because this interaction is ``important'' for both
species) (Cazelles et al. 2016), but here we show increasing strength of
associations leads to increasing probability of false-negatives in
interaction data. Further observed co-occurrence is often equated with
meaningful interaction strength, but this is not necessarily the case
(Blanchet, Cazelles, and Gravel 2020; Strydom et al. 2021). Bears and
salmon \emph{interact}---a bear and the microbes in the soil of a dens
interact, but less so.

\begin{figure}
\hypertarget{fig:taxonomy}{%
\centering
\includegraphics{./figures/concept_v3.png}
\caption{Taxonomy of false-negatives in data for two hypothetical
species A and B, where in reality A and B do interact in some
capacity.}\label{fig:taxonomy}
}
\end{figure}

Here, we show that the probability of observing a actual
``non-interaction'' between species depends on sampling effort, and
suggest that surveys of species interactions can benefit from simulation
modeling of detection probability (Jordano 2016b). We demonstrate that
the realized false-negative rate of interactions is directly related the
relative abundance of a particular species, relationship between total
sampling effort (the total count of all individuals of all species seen)
and false-negative rate. questions we pose and attempt to answer are: 1)
How many times do you have to observe a non-interaction between two
species to be confident in saying that is a true negative? 2) How
``wrong'' are the measurements of network structure as a function of
false-negative probability? and lastly 3) How do false-negatives impact
our ability to make reliable predictions about interactions? We show
that positive associations in co-occurrence data can increase realized
probability of false negatives, and demonstrate these positive
associations are present in two spatially-replicated systems. We
conclude by suggesting that simulation of sampling effort and species
occurrence can and should be used to help design surveys of species
diversity (Moore and McCarthy 2016), and by advocating use of null
models like those presented here as a tool for guiding design of surveys
of species interactions, and for modeling detection error in predictive
ecological models.

\hypertarget{how-many-observations-of-a-non-interaction-do-we-need-to-classify-it-as-a-true-negative}{%
\section{How many observations of a non-interaction do we need to
classify it as a true
negative?}\label{how-many-observations-of-a-non-interaction-do-we-need-to-classify-it-as-a-true-negative}}

To answer the titular question of this section, we present a naive model
of interaction detection: we assume that every interacting pair of
species is incorrectly observed as a not-interacting with an independent
and fixed probability, which we denote \(p_{fn}\) and subsequently refer
to as the False-Negative Rate (FNR). If we observe the same species
not-interacting \(N\) times, then the probability of a true-negative
(denoted \(p_{tn}\)) is given by \(p_{tn} = 1 - (p_{fn})^N\). This
relation (a special case of the negative-binomial distribution) is shown
in fig.~\ref{fig:negativebinom} for varying values of the false negative
rate \(p_{fn}\). This illustrates a fundamental link between our ability
to reliably say an interaction doesn't exist---\(p_{tn}\)---and the
number of times we have observed a given species. In addition, note that
there also is no non-zero \(p_{fn}\) for which we can ever \emph{prove}
that an interaction does not exist---no matter how many observations of
non-interaction \(N\) we have, \(p_{tn} < 1\).

\begin{figure}
\hypertarget{fig:negativebinom}{%
\centering
\includegraphics{./figures/negativebinomial.png}
\caption{The probability an observed interaction is a ``true negative''
(y-axis) given how many times it has been sampled as a non-interaction
(x-axis). Each color reflects a different value of \(p_{fn}\), the
false-negative rate (FNR). This is effectively the cdf of the
negative-binomial distribution with \(r=1\). It's the birthday paradox,
but backwards.}\label{fig:negativebinom}
}
\end{figure}

From fig.~\ref{fig:negativebinom} (and general intuition) it is clear
that the more times we see two species \emph{occurring}, but \emph{not}
interacting, the more likely the interaction is a true negative. But how
does one decide what this threshold of number of observations should be
when planning to sample a given system? If false-negative rates
presented in fig.~\ref{fig:negativebinom} seem unrealistically high,
consider that species are not observed independent of their relative
abundance. In the next section we demonstrate that distribution of
abundance in ecosystems can lead to realized values of \(p_{fn}\)
similar to those in fig.~\ref{fig:negativebinom} for species with low
relative abundance, simply as a artifact of sampling effort.

\hypertarget{false-negatives-as-a-product-of-relative-abundance}{%
\subsection{False-negatives as a product of relative
abundance}\label{false-negatives-as-a-product-of-relative-abundance}}

Here we show the realized false-negative rate of species interactions
changes drastically with sampling effort, largely due to the intrinsic
variation of abundances within a community. We do this by simulating the
process of observation of species interactions, applied both to 243
empirical food webs from the Mangal database (Banville, Vissault, and
Poisot 2021) as well as random food-webs generated using the niche model
(Richard J. Williams and Martinez 2000). Our neutral model of
observation assumes each observed species is drawn from the distribution
of those species' abundances at that place and time. Although there is
no shortage of debate as to the processes the govern this distribution
of abundances within a community, this abundance distribution can be
reasonably-well described by a log-normal distribution (Volkov et al.
2003) (Note that in addition to the log-normal distribution, we also
tested the case where the abundance distribution is derived from
power-law scaling \(Z^{(T_i-1)}\) where \(T_i\) is the trophic level of
species \(i\) and \(Z\) is a scaling coefficient. (Savage et al. 2004),
which yields the same qualitative behavior, \emph{supplement figure 1}).
The practical consequence of this skewed distribution of biomass in
communities is seeing two low biomass species interacting requires two
low probability events: observing two species of low relative biomass
\emph{at the same time}.

To simulate the process of observation, for an ecological network \(A\)
with \(S\) species, we sample abundances for each species from a
standard-log-normal distribution. For each true interaction in \(A\)
(i.e.~\(A_{ij} = 1\)) we estimate the probability of observing both
species \(i\) and \(j\) at given place and time by simulating \(n\)
observations of individuals, where the species of the individual
observed at the \(1,2,\dots,n\)-th observation is drawn from the
generated log-normal distribution of abundances. For each pair of
species \((i,j)\), if both \(i\) and \(j\) are observed within the \(n\)
observations, the interaction is tallied as a true positive if
\(A_{ij}=1\) and a false positive otherwise. Similarly, if only one of
\(i\) and \(j\) are observed---\emph{but not both}---in these \(n\)
observations, but \(A_{ij}=1\), this is counted as a false-negative, and
a true-negative otherwise.

In fig.~\ref{fig:totalobs} (a) we see this model of observation applied
to networks generated using the niche model (Richard J. Williams and
Martinez 2000) across varying levels of species richness, and in (b)
applied to 243 food-webs from the Mangal database. For all niche model
simulations in this manuscript, for a given number of species \(S\) the
number of interactions is drawn from the flexible-links model fit to
Mangal data (MacDonald, Banville, and Poisot 2020), effectively drawing
the number of interactions \(L\) for a random niche model food-web as
\(L \sim \text{BetaBinomial}(S^2-S+1, \mu \phi, (1-\mu)\phi)\), where
the MAP estimate of (\(\mu\), \(\phi\)) applied to Mangal data from
MacDonald, Banville, and Poisot (2020) is \((\mu = 0.086, \phi =24.3)\).
All simulations were done with 500 independent replicates per unique
number of observations \(n\). All analyses presented here are done in
Julia v1.6 (Bezanson et al. 2015) using both EcologicalNetworks.jl v0.5
and Mangal.jl v0.4 {[}Banville, Vissault, and Poisot (2021); ZENODO link
TODO{]}. Note that the empirical data also is, due to the phenomena
described here, very likely to \emph{already} have many false negatives,
which is why we are interested in prediction of networks in the first
place---we'll revisit this in the final section.

\begin{figure}
\hypertarget{fig:totalobs}{%
\centering
\includegraphics{./figures/combinedfig2.png}
\caption{A and B: False negative rate (y-axis) as a function of total
sampling effort (x-axis) and network size, computed using the method
described above. For 500 independent draws from the niche model (Richard
J. Williams and Martinez 2000) at varying levels of species richness
(colors) with connectance drawn according to the flexible-links model
(MacDonald, Banville, and Poisot 2020) as described in the main text.
For each draw from the niche model, 200 sets of 1500 observations are
simulated, for which each the mean false negative rate at each
observation-step is computed. Means denoted with points, with
\(1\sigma\) in the first shade and \(2\sigma\) in the second. B:
empirical food webs from Mangal database in teal, applied to the same
process as the A. The outlier on panel B is a 714 species food-web. C)
The expected needed observations of all individuals of all species
(y-axis) required to obtain a goal number of observations (colors) of a
particular species, and a function of the relative abundance of that
focal species (x-axis)}\label{fig:totalobs}
}
\end{figure}

In panel (c) of fig.~\ref{fig:totalobs}, we show the expected number of
total observations needed to obtain a ``goal'' number of observations
(colors) of a particular ``focal'' species. As an example, if we
hypothesize that \(A\) and \(B\) do not interact, and we want to see
species \(A\) and \(B\) both co-occurring and not-interacting 10 times
to be confident this is a negative (a la fig.~\ref{fig:negativebinom}),
then we need an expected 10,000 observations of all species if the
relative abundance of \(A\) is 0.00125.

Empirical data on interactions are subject to the practical limitations
of funding and human-work hours, and therefore existing data tend to
fall on the order on 100s or 1000s observations of individuals per site
(Resasco, Chacoff, and Vázquez 2021; Schwarz et al. 2020; Nielsen and
Bascompte 2007). Clear aggregation of this data has proven difficult to
find and a meta-analysis of network data and sampling effort seems both
pertinent and necessary, in addition to the effects of aggregation of
interactions across taxonomic scales (Giacomuzzo and Jordán 2021;
Gauzens et al. 2013). Further, from fig.~\ref{fig:totalobs} it is
evident that the number of species considered in a study is inseparable
from the false-negative rate in that study, and this effect should be
taken into account when designing samples of ecological networks in the
future.

We conclude this section by advocating for the use of neutral models
similar to above to generate expectations about the number of
false-negatives in a data set of a given size. This could prove fruitful
both for designing surveys of interactions (Canard et al. 2012), but
also because we may want to incorporate models of observation error into
predictive models (Joseph 2020). Additionally, one must consider the
context for sampling---is the goal to detect a particular species \(A\)
(as in fig.~\ref{fig:totalobs} (c)), or to get a representative sample
of interactions across the species pool? This argument is
well-considered when sampling species (Willott 2001), but has not yet
been internalized for designing samples of communities.

\hypertarget{positive-associations-can-increase-the-probability-of-false-negatives}{%
\subsection{Positive associations can increase the probability of
false-negatives}\label{positive-associations-can-increase-the-probability-of-false-negatives}}

This model above doesn't consider the possibility that there are
positive or negative associations which shift the probability of
observing two species together due to their interaction (Cazelles et al.
2016). However, here we demonstrate that the probability of observing a
false negative can be \emph{higher} if there is some positive
association between occurrence of species \(A\) and \(B\).

If we denote the probability that we observe an interaction we know
exists between \(A\) and \(B\) as \(P(AB)\), and if there is \emph{no}
association between the marginal probabilities of observing \(A\) and
observing \(B\), denoted \(P(A)\) and \(P(B)\) respectively, then the
probability of observing the interaction \(P(AB) = P(A)P(B)\). In the
other case where there \emph{is} some positive strength of association
between observing both \(A\) and \(B\) because this interaction is
``important'' for each species, then the probability of observation both
\(A\) and \(B\), \(P(AB)\), is greater than \(P(A)P(B)\) as \(P(A)\) and
\(P(B)\) are not independent and instead are positively correlated,
\emph{i.e.} \(P(AB) > P(A)P(B)\). In this case, the probability of
observing a false negative in our naive model from
fig.~\ref{fig:negativebinom} is \(p_{fn} = 1 - P(AB)\) which due to the
above inequality implies \(p_{fn} \geq 1 - P(A)P(B)\) which indicates
increasingly greater probability of a false negative as
\(P(AB) \to P(AB) \gg P(A)P(B)\).

However this does not consider variation in species abundance in space
and time, (Poisot, Stouffer, and Gravel 2015). If positive or negative
associations between species structure variation in the distribution of
\(P(AB)\) across space/time, then the spatial/temporal biases induced by
data collection would further impact the realized false negative rate,
as the probability of false negative would not be constant for each pair
of species across sites. To test for this association empirical data, we
use two datasets: a set of host-parasite interactions sampled across 51
sites with 327 total taxa (Hadfield et al. 2014) and a set of 18 New
Zealand freshwater stream food webs with 566 total taxa (R. M. Thompson
and Townsend 2000). We simply compute the empirical marginal
distribution of species occurrence, and compare the product of the
marginals, \(P(A)P(B)\), to the empirical joint distribution \(P(AB)\).

\begin{figure}
\hypertarget{fig:associations}{%
\centering
\includegraphics{./figures/positiveassociations.png}
\caption{Top: Hadfield, Bottom: NZ Stream Foodwebs. Effectively a
version of Cazelles et al. (2016) figure 1 panel A. Both distributions
have \(\mu \neq 0\) with \(p < 10^{-50}\)}\label{fig:associations}
}
\end{figure}

In fig.~\ref{fig:associations}, both host-parasite system (top) and
food-web (bottom) exhibit these positive associations. There is no
reason to expect the strength of this association to be the same in
different systems. At the moment, computing this metric for all of the
networks in the Mangal database proves challenging as most data sets use
different taxonmic identifiers, often at different resolutions. These
particular datasets (Hadfield et al. 2014; R. M. Thompson and Townsend
2000) were usable because they already have been sorted to have a fixed
taxonomic backbone (as part of EcologicalNetworks.jl (Banville,
Vissault, and Poisot 2021)). Applying this in bulk to Mangal food-webs
presents the difficulty of resolving different taxon identifiers across
spatial samples of species with to different resolutions, which is why
we can't simply apply this to the whole Mangal database---this
highlights a general problem of resolving taxonomic indentifiers which
use different names and different resolutions in different ecological
datasets, which is a problem that needs to be addressed for
computational approaches to scale up to the world of big-ecological-data
we hope to build, although this is a task that may be aided via
natural-language-processing methods.

\hypertarget{the-impact-of-false-negatives-on-network-analysis-and-prediction}{%
\section{The impact of false-negatives on network analysis and
prediction}\label{the-impact-of-false-negatives-on-network-analysis-and-prediction}}

We now transition toward assessing the effects of false negatives in
data on the properties of the networks which we derive from this
interaction data, and their effect on models for predicting interactions
in the future.

\hypertarget{effects-of-false-negatives-on-network-properties}{%
\subsection{Effects of false-negatives on network
properties}\label{effects-of-false-negatives-on-network-properties}}

Here we simulate the process of observation with error to generate
synthetic data with a known proportion of false negatives, and compare
the computed network properties of the original ``true'' network to the
computed properties of the ``observed'' network with added
false-negatives. In fig.~\ref{fig:properties} we see the mean-squared
error of connectance, mean degree-centrality, and spectral radius,
computed across 2000, 2000, and 300 replicates respectively at each
value of the false negative rate \(p_{fn}\). All replicates use random
food-webs simulated using the niche model (Richard J. Williams and
Martinez 2000) with \(100\) species and connectance drawn from the
flexible-links model (MacDonald, Banville, and Poisot 2020) as before.

\begin{figure}
\hypertarget{fig:properties}{%
\centering
\includegraphics{./figures/props_specrad.png}
\caption{The mean-squared error (y-axis) of various network properties
(different colors) across various simulated false-negative rates
(x-axis). Means denoted with points, with \(1\sigma\) in the first shade
and \(2\sigma\) in the second.}\label{fig:properties}
}
\end{figure}

We consider three properties: connectance, mean-degree-centrality, and
spectral radius, indicative of local, meso, and global structure.
Connectance is effectively a node-level property, a proxy for the degree
distribution. Degree-centrality captures a different aspect of network
structure than connectance, more indicative of meso-level properties
that describe local `regions' of nodes interact. Spectral radius
(equivalent to the magnitude of the largest eigenvalue of \(A\)) is a
measure of global structure, and demonstrates the most variability in
response to false-negatives. For example, if a false-negative splits a
metaweb into two components, spectral-radius becomes the largest
eigenvalue of each of those two components. Also note that the form of
this error function varies little as species richness changes
(\emph{supplemental figure 2}). Practically, fig.~\ref{fig:properties}
shows us that different scales of measuring network structure vary in
their response to false negatives---connectance responds roughly
linearly to false negatives, whereas mean-degree-centrality decisively
does not. This implies that false-negatives adversely could effect
indirect interactions (R. J. Williams et al. 2002).

\hypertarget{effects-of-false-negatives-on-ability-to-make-predictions}{%
\subsection{Effects of false negatives on ability to make
predictions}\label{effects-of-false-negatives-on-ability-to-make-predictions}}

Here, we assess the effect of false negatives in data on our ability to
make predictions about interactions. The prevalence of false-negatives
in data is the catalyst for interaction prediction in the first place,
and as a result methods have been proposed to counteract this bias
(Poisot, Ouellet, et al. 2021; Stock et al. 2017). However, it is
feasible this could induce too much noise for a interaction prediction
model to detect the signal of interaction chance from to the latent
properties of each species derived from the empirical network if the
number of false-negatives in a dataset becomes too overwhelming.

To test this, we use the same predictive model and dataset as in Strydom
et al. (2021) to predict a metaweb from various empirical slices of the
species pool observed across space. This dataset from Hadfield et al.
(2014) describes host-parasite interaction networks sampled across 51
sites. We partition the data into 80-20 training-test split, and then
seed the training data with false negatives varying rates, but crucially
do nothing to the test data. We use the same model, a neural-network
with 3 feed-forward layers to predict outputs based on features
extracted from co-occurence (see Strydom et al. (2021) for more
details). The single modification we make to the model is not enforcing
a number of positives in the training data as this constraint is
eventually impossible for increasing FNR. In fig.~\ref{fig:rocpr}, we
show receiving-operating-characteristic (ROC) and precision-recall (PR)
curves for the model with varying levels of synthetic false-negatives
added to the data.

\begin{figure}
\hypertarget{fig:rocpr}{%
\centering
\includegraphics{./figures/rocpr_falsenegatives.png}
\caption{Receiver-operating-characteristic (left) and precision-recall
(right) curves for the model on varying levels of false-negatives in the
data (colors). For each value of FNR, we run 30 random training/test
splits on 80/20 percent of the data. Replica of figure 1 in Strydom et
al. (2021)}\label{fig:rocpr}
}
\end{figure}

Interestingly, the performance of the model from Strydom et al. (2021)
changes little with many added false-negatives, which is good evidence
in favor neural-networks as a class of model for interaction detection.
Again, similar to our caveat in the previous section, this dazta is
\emph{already} likely to have many false-negatives, so the effects of
adding more as we do in this illustration might be mitigated because
there are already non-simulated false-negatives in the original data
which impact the models performance, even in the \(p_{fn} = 0\) case.

We conclude be proposing that simulating the effects of false negatives
in this way can serve as an additional validation tool when aiming to
detect structural properties of networks using generative null models
(Connor, Barberán, and Clauset 2017), or when evaluating the robustness
of a predictive model.

\hypertarget{discussion}{%
\section{Discussion}\label{discussion}}

Here, we have demonstrated that we expect false-negatives in species
interaction datasets purely due to the distribution of abundances within
a community. Positive associations between species occurrence (Cazelles
et al. 2016) can increase the realized false-negative rate if the
sampling effort is limited, and we have presented evidence of this
non-random structure of co-occurrence in two sets of
spatially-replicated ecological network samples. We have also shown that
false-negatives can cause varying responses in our measurements of
network properties and further could impact our ability to reliably
predict interactions, which highlights the need for further research
into methods for correcting this bias in existing data (Stock et al.
2017). A brief caveat here is that we do not consider the rate of
false-positives---in large part false-positives can be explained by
misidentification of species, although this could be a relevant
consideration in some cases.

What does the future hold for this research? A better understanding of
how false-negatives impact our analyses and prediction of ecological
networks is a practical necessity. False-negatives could pose a problem
for many forms of inference in network ecology. For example, if we aim
to measure structural or dynamic stability of a network, or to infer
indirect interactions (R. J. Williams et al. 2002), these estimates
could be prone to error if the observed network is not sampled
``enough.'' What exactly ``enough'' means is then specific to the
application, and should be assessed via methods like those here when
designing samples. Further, predictions about network rewiring (P. L.
Thompson and Gonzalez 2017) due to a changing climate could be
error-prone without accounting for interactions that have not been
observed but that still may become climatically infeasible.

This highlights the need for a quantitatively robust approach samples
design: for interactions (Jordano 2016b) and otherwise (Carlson et al.
2020). The primary takeaway is that when planning the sampling effort
across sites, it is necessary to take both the size of the species pool
into account. Further, simulating the process of observation could be a
powerful tool for planing study design which takes relative abundance
into account, and provide a null baseline for detection of interaction
strength. A model similar to that here can and should be used to provide
a neutral expectation of true-negative probability given a number of
observations of individuals at a given place and time.

As we derive from fig.~\ref{fig:negativebinom}, we can never guarantee
there are no false-negatives in data. In recent years, there has been
interest toward explicitly accounting for false-negatives in models
(Young, Valdovinos, and Newman 2021; Stock et al. 2017), and toward a
predictive approach toward interactions ---rather than expect that our
samples can fully capture all interactions, we know that some
interactions between species will not be observed due to finite sampling
capacity, and instead we must impute the true metaweb of interactions
given a set of samples (Strydom et al. 2021). As a result, better
predictive approaches are needed for interaction networks (Strydom et
al. 2021), and building models that explicitly account for observation
error is a necessary step forward for predictive ecological models
(Young, Valdovinos, and Newman 2021; Johnson and Larremore 2021). Neural
networks, like the one used to predict interactions in the above
section, have been used to reflect hidden states which account for
detection error in occupancy modeling (Joseph 2020), and could be
integrated in the predictive models of the future.

A better conceptual framework for designing surveys and monitoring
networks, and incorporating sequential observations over time is clearly
needed (Carlson et al. 2020), combined with a meta-analysis of sampling
effort and taxonomic resolution in existing data. Incorporating a better
understanding of sampling effects and bias on both the future design of
biodiversity monitoring systems, and the predictive models we wish to
apply to this data, is imperative in making actionable predictions about
the future of ecological interactions on our planet.

\hypertarget{references}{%
\section*{References}\label{references}}
\addcontentsline{toc}{section}{References}

\hypertarget{refs}{}
\begin{CSLReferences}{1}{0}
\leavevmode\hypertarget{ref-deAguiar2019RevBia}{}%
Aguiar, Marcus A. M. de, Erica A. Newman, Mathias M. Pires, Justin D.
Yeakel, Carl Boettiger, Laura A. Burkle, Dominique Gravel, et al. 2019.
{``Revealing Biases in the Sampling of Ecological Interaction
Networks.''} \emph{PeerJ} 7 (September): e7566.
\url{https://doi.org/10.7717/peerj.7566}.

\leavevmode\hypertarget{ref-Banville2021ManJl}{}%
Banville, Francis, Steve Vissault, and Timothée Poisot. 2021.
{``Mangal.jl and EcologicalNetworks.jl: Two Complementary Packages for
Analyzing Ecological Networks in Julia.''} \emph{Journal of Open Source
Software} 6 (61): 2721. \url{https://doi.org/10.21105/joss.02721}.

\leavevmode\hypertarget{ref-Becker2021OptPre}{}%
Becker, Daniel J., Gregory F. Albery, Anna R. Sjodin, Timothée Poisot,
Laura M. Bergner, Tad A. Dallas, Evan A. Eskew, et al. 2021.
{``Optimizing Predictive Models to Prioritize Viral Discovery in
Zoonotic Reservoirs.''} \emph{bioRxiv}, August, 2020.05.22.111344.
\url{https://doi.org/10.1101/2020.05.22.111344}.

\leavevmode\hypertarget{ref-Bezanson2015JulFre}{}%
Bezanson, Jeff, Alan Edelman, Stefan Karpinski, and Viral B. Shah. 2015.
{``Julia: A Fresh Approach to Numerical Computing.''}
\emph{arXiv:1411.1607 {[}cs{]}}, July.
\url{http://arxiv.org/abs/1411.1607}.

\leavevmode\hypertarget{ref-Blanchet2020CooNot}{}%
Blanchet, F. Guillaume, Kevin Cazelles, and Dominique Gravel. 2020.
{``Co-Occurrence Is Not Evidence of Ecological Interactions.''}
\emph{Ecology Letters} 23 (7): 1050--63.
\url{https://doi.org/10.1111/ele.13525}.

\leavevmode\hypertarget{ref-Boakes2015InfSpe}{}%
Boakes, Elizabeth H., Tracy M. Rout, and Ben Collen. 2015. {``Inferring
Species Extinction: The Use of Sighting Records.''} \emph{Methods in
Ecology and Evolution} 6 (6): 678--87.
\url{https://doi.org/10.1111/2041-210X.12365}.

\leavevmode\hypertarget{ref-Canard2012EmeStr}{}%
Canard, Elsa, Nicolas Mouquet, Lucile Marescot, Kevin J. Gaston,
Dominique Gravel, and David Mouillot. 2012. {``Emergence of Structural
Patterns in Neutral Trophic Networks.''} \emph{PLOS ONE} 7 (8): e38295.
\url{https://doi.org/10.1371/journal.pone.0038295}.

\leavevmode\hypertarget{ref-Carlson2020WhaWou}{}%
Carlson, Colin J., Tad A. Dallas, Laura W. Alexander, Alexandra L.
Phelan, and Anna J. Phillips. 2020. {``What Would It Take to Describe
the Global Diversity of Parasites?''} \emph{Proceedings of the Royal
Society B: Biological Sciences} 287 (1939): 20201841.
\url{https://doi.org/10.1098/rspb.2020.1841}.

\leavevmode\hypertarget{ref-Cazelles2016TheSpe}{}%
Cazelles, Kévin, Miguel B. Araújo, Nicolas Mouquet, and Dominique
Gravel. 2016. {``A Theory for Species Co-Occurrence in Interaction
Networks.''} \emph{Theoretical Ecology} 9 (1): 39--48.
\url{https://doi.org/10.1007/s12080-015-0281-9}.

\leavevmode\hypertarget{ref-Connor2017UsiNul}{}%
Connor, Nora, Albert Barberán, and Aaron Clauset. 2017. {``Using Null
Models to Infer Microbial Co-Occurrence Networks.''} \emph{PLOS ONE} 12
(5): e0176751. \url{https://doi.org/10.1371/journal.pone.0176751}.

\leavevmode\hypertarget{ref-Gauzens2013FooAgg}{}%
Gauzens, Benoît, Stéphane Legendre, Xavier Lazzaro, and Gérard Lacroix.
2013. {``Food-Web Aggregation, Methodological and Functional Issues.''}
\emph{Oikos} 122 (11): 1606--15.

\leavevmode\hypertarget{ref-Giacomuzzo2021FooWeb}{}%
Giacomuzzo, Emanuele, and Ferenc Jordán. 2021. {``Food Web Aggregation:
Effects on Key Positions.''} Preprint. Ecology.
\url{https://doi.org/10.1101/2021.04.18.440319}.

\leavevmode\hypertarget{ref-Griffiths1998SamEff}{}%
Griffiths, David. 1998. {``Sampling Effort, Regression Method, and the
Shape and Slope of Sizeabundance Relations.''} \emph{Journal of Animal
Ecology} 67 (5): 795--804.
\url{https://doi.org/10.1046/j.1365-2656.1998.00244.x}.

\leavevmode\hypertarget{ref-Hadfield2014TalTwo}{}%
Hadfield, Jarrod D., Boris R. Krasnov, Robert Poulin, and Shinichi
Nakagawa. 2014. {``A Tale of Two Phylogenies: Comparative Analyses of
Ecological Interactions.''} \emph{The American Naturalist} 183 (2):
174--87. \url{https://doi.org/10.1086/674445}.

\leavevmode\hypertarget{ref-Johnson2021BayEst}{}%
Johnson, Erik K., and Daniel B. Larremore. 2021. {``Bayesian Estimation
of Population Size and Overlap from Random Subsamples.''}
\emph{bioRxiv}, July, 2021.07.06.451319.
\url{https://doi.org/10.1101/2021.07.06.451319}.

\leavevmode\hypertarget{ref-Jordano2016ChaEco}{}%
Jordano, Pedro. 2016a. {``Chasing Ecological Interactions.''} \emph{PLOS
Biology} 14 (9): e1002559.
\url{https://doi.org/10.1371/journal.pbio.1002559}.

\leavevmode\hypertarget{ref-Jordano2016SamNet}{}%
---------. 2016b. {``Sampling Networks of Ecological Interactions.''}
\emph{Functional Ecology}, September.
\url{https://doi.org/10.1111/1365-2435.12763}.

\leavevmode\hypertarget{ref-Joseph2020NeuHie}{}%
Joseph, Maxwell B. 2020. {``Neural Hierarchical Models of Ecological
Populations.''} \emph{Ecology Letters} 23 (4): 734--47.
\url{https://doi.org/10.1111/ele.13462}.

\leavevmode\hypertarget{ref-Kenall2014OpeFut}{}%
Kenall, Amye, Simon Harold, and Christopher Foote. 2014. {``An Open
Future for Ecological and Evolutionary Data?''} \emph{BMC Evolutionary
Biology} 14 (1): 66. \url{https://doi.org/10.1186/1471-2148-14-66}.

\leavevmode\hypertarget{ref-MacDonald2020RevLin}{}%
MacDonald, Arthur Andrew Meahan, Francis Banville, and Timothée Poisot.
2020. {``Revisiting the Links-Species Scaling Relationship in Food
Webs.''} \emph{Patterns} 1 (7).
\url{https://doi.org/10.1016/j.patter.2020.100079}.

\leavevmode\hypertarget{ref-Makiola2020KeyQue}{}%
Makiola, Andreas, Zacchaeus G. Compson, Donald J. Baird, Matthew A.
Barnes, Sam P. Boerlijst, Agnès Bouchez, Georgina Brennan, et al. 2020.
{``Key Questions for Next-Generation Biomonitoring.''} \emph{Frontiers
in Environmental Science} 7.
\url{https://doi.org/10.3389/fenvs.2019.00197}.

\leavevmode\hypertarget{ref-Martinez1999EffSam}{}%
Martinez, Neo D., Bradford A. Hawkins, Hassan Ali Dawah, and Brian P.
Feifarek. 1999. {``Effects of Sampling Effort on Characterization of
Food-Web Structure.''} \emph{Ecology} 80 (3): 1044--55.
\url{https://doi.org/10.1890/0012-9658(1999)080\%5B1044:EOSEOC\%5D2.0.CO;2}.

\leavevmode\hypertarget{ref-Moore2016OptEco}{}%
Moore, Alana L., and Michael A. McCarthy. 2016. {``Optimizing Ecological
Survey Effort over Space and Time.''} \emph{Methods in Ecology and
Evolution} 7 (8): 891--99.
\url{https://doi.org/10.1111/2041-210X.12564}.

\leavevmode\hypertarget{ref-Nielsen2007EcoNet}{}%
Nielsen, Anders, and Jordi Bascompte. 2007. {``Ecological Networks,
Nestedness and Sampling Effort.''} \emph{Journal of Ecology} 95 (5):
1134--41. \url{https://doi.org/10.1111/j.1365-2745.2007.01271.x}.

\leavevmode\hypertarget{ref-Paine1988RoaMap}{}%
Paine, R. T. 1988. {``Road Maps of Interactions or Grist for Theoretical
Development?''} \emph{Ecology} 69 (6): 1648--54.
\url{https://doi.org/10.2307/1941141}.

\leavevmode\hypertarget{ref-Poisot2021GloKno}{}%
Poisot, Timothée, Gabriel Bergeron, Kevin Cazelles, Tad Dallas,
Dominique Gravel, Andrew MacDonald, Benjamin Mercier, Clément Violet,
and Steve Vissault. 2021. {``Global Knowledge Gaps in Species
Interaction Networks Data.''} \emph{Journal of Biogeography}, April,
jbi.14127. \url{https://doi.org/10.1111/jbi.14127}.

\leavevmode\hypertarget{ref-Poisot2021ImpMam}{}%
Poisot, Timothée, Marie-Andrée Ouellet, Nardus Mollentze, Maxwell J.
Farrell, Daniel J. Becker, Gregory F. Albery, Rory J. Gibb, Stephanie N.
Seifert, and Colin J. Carlson. 2021. {``Imputing the Mammalian Virome
with Linear Filtering and Singular Value Decomposition.''}
\emph{arXiv:2105.14973 {[}q-Bio{]}}, May.
\url{http://arxiv.org/abs/2105.14973}.

\leavevmode\hypertarget{ref-Poisot2015SpeWhy}{}%
Poisot, Timothée, Daniel B. Stouffer, and Dominique Gravel. 2015.
{``Beyond Species: Why Ecological Interaction Networks Vary Through
Space and Time.''} \emph{Oikos} 124 (3): 243--51.
\url{https://doi.org/10.1111/oik.01719}.

\leavevmode\hypertarget{ref-Resasco2021PlaPol}{}%
Resasco, Julian, Natacha P. Chacoff, and Diego P. Vázquez. 2021.
{``Plantpollinator Interactions Between Generalists Persist over Time
and Space.''} \emph{Ecology} 102 (6): e03359.
\url{https://doi.org/10.1002/ecy.3359}.

\leavevmode\hypertarget{ref-Savage2004EffBod}{}%
Savage, Van M., James F. Gillooly, James H. Brown, Geoffrey B. West, and
Eric L. Charnov. 2004. {``Effects of Body Size and Temperature on
Population Growth.''} \emph{The American Naturalist} 163 (3): 429--41.
\url{https://doi.org/10.1086/381872}.

\leavevmode\hypertarget{ref-Schwarz2020TemSca}{}%
Schwarz, Benjamin, Diego P. Vázquez, Paul J. CaraDonna, Tiffany M.
Knight, Gita Benadi, Carsten F. Dormann, Benoit Gauzens, et al. 2020.
{``Temporal Scale-Dependence of Plantpollinator Networks.''}
\emph{Oikos} 129 (9): 1289--1302.
\url{https://doi.org/10.1111/oik.07303}.

\leavevmode\hypertarget{ref-Stephenson2020TecAdv}{}%
Stephenson, PJ. 2020. {``Technological Advances in Biodiversity
Monitoring: Applicability, Opportunities and Challenges.''}
\emph{Current Opinion in Environmental Sustainability} 45 (August):
36--41. \url{https://doi.org/10.1016/j.cosust.2020.08.005}.

\leavevmode\hypertarget{ref-Stock2017LinFil}{}%
Stock, Michiel, Timothée Poisot, Willem Waegeman, and Bernard De Baets.
2017. {``Linear Filtering Reveals False Negatives in Species Interaction
Data.''} \emph{Scientific Reports} 7 (1): 45908.
\url{https://doi.org/10.1038/srep45908}.

\leavevmode\hypertarget{ref-Strydom2021RoaPre}{}%
Strydom, Tanya, Michael David Catchen, Francis Banville, Dominique
Caron, Gabriel Dansereau, Philippe Desjardins-Proulx, Norma Forero, et
al. 2021. {``A Roadmap Toward Predicting Species Interaction Networks
(Across Space and Time).''} Preprint. EcoEvoRxiv.
\url{https://doi.org/10.32942/osf.io/eu7k3}.

\leavevmode\hypertarget{ref-Thompson2017DisGov}{}%
Thompson, Patrick L., and Andrew Gonzalez. 2017. {``Dispersal Governs
the Reorganization of Ecological Networks Under Environmental Change.''}
\emph{Nature Ecology \& Evolution} 1 (6): 1--8.
\url{https://doi.org/10.1038/s41559-017-0162}.

\leavevmode\hypertarget{ref-Thompson2000ResSol}{}%
Thompson, Ross M., and Colin R. Townsend. 2000. {``Is Resolution the
Solution?: The Effect of Taxonomic Resolution on the Calculated
Properties of Three Stream Food Webs.''} \emph{Freshwater Biology} 44
(3): 413--22. \url{https://doi.org/10.1046/j.1365-2427.2000.00579.x}.

\leavevmode\hypertarget{ref-Volkov2003NeuThe}{}%
Volkov, Igor, Jayanth R. Banavar, Stephen P. Hubbell, and Amos Maritan.
2003. {``Neutral Theory and Relative Species Abundance in Ecology.''}
\emph{Nature} 424 (6952): 1035--37.
\url{https://doi.org/10.1038/nature01883}.

\leavevmode\hypertarget{ref-Walther1995SamEff}{}%
Walther, B. A., P. Cotgreave, R. D. Price, R. D. Gregory, and D. H.
Clayton. 1995. {``Sampling Effort and Parasite Species Richness.''}
\emph{Parasitology Today} 11 (8): 306--10.
\url{https://doi.org/10.1016/0169-4758(95)80047-6}.

\leavevmode\hypertarget{ref-Williams2002TwoDeg}{}%
Williams, R. J., E. L. Berlow, J. A. Dunne, A.-L. Barabasi, and N. D.
Martinez. 2002. {``Two Degrees of Separation in Complex Food Webs.''}
\emph{Proceedings of the National Academy of Sciences} 99 (20):
12913--16. \url{https://doi.org/10.1073/pnas.192448799}.

\leavevmode\hypertarget{ref-Williams2000SimRul}{}%
Williams, Richard J., and Neo D. Martinez. 2000. {``Simple Rules Yield
Complex Food Webs.''} \emph{Nature} 404 (6774): 180--83.
\url{https://doi.org/10.1038/35004572}.

\leavevmode\hypertarget{ref-Willott2001SpeAcc}{}%
Willott, S. J. 2001. {``Species Accumulation Curves and the Measure of
Sampling Effort.''} \emph{Journal of Applied Ecology} 38 (2): 484--86.
\url{https://doi.org/10.1046/j.1365-2664.2001.00589.x}.

\leavevmode\hypertarget{ref-Young2021RecPla}{}%
Young, Jean-Gabriel, Fernanda S. Valdovinos, and M. E. J. Newman. 2021.
{``Reconstruction of Plantpollinator Networks from Observational
Data.''} \emph{Nature Communications} 12 (1): 3911.
\url{https://doi.org/10.1038/s41467-021-24149-x}.

\end{CSLReferences}

\end{document}
